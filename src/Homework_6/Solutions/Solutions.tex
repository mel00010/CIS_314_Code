\documentclass[12pt]{article}
% Useful packages that enable things like math symbols and a fancy header

\usepackage{tikz}
\usetikzlibrary{arrows,shapes.gates.logic.US,shapes.gates.logic.IEC,calc}
\usepackage{amssymb,amsfonts,amsmath,amsthm}
\usepackage{latexsym}
\usepackage{mathrsfs}
\usepackage{fancyhdr}
\usepackage{outlines}
\usepackage{enumitem}
\usepackage{float}
\usepackage{calc}


\tikzset{
    between/.style args={#1 and #2}{
         at = ($(#1)!0.5!(#2)$)
    }
}

\tikzset{
    between base/.style args={#1 and #2}{
        between=#1.base and #2.base
    }
}


% Uses the package declared above and makes the page have headers and footers
\pagestyle{fancy}

% Puts a title at the top left
\lhead{CIS 314}
% Puts a title at the center
\chead{Assignment 5}

% Puts a title at the top right
\rhead{Mel McCalla}

% Put's "today's date" in the lower left with the command \today
\lfoot{\today}
\cfoot{\thepage}

%%%%  Some user defined commands that make life easier.  
\newcommand{\be}{\begin{equation}}
\newcommand{\ee}{\end{equation}}
\newcommand{\bea}{\begin{eqnarray}}
\newcommand{\eea}{\end{eqnarray}}
\newcommand{\beas}{\begin{eqnarray*}}
\newcommand{\eeas}{\end{eqnarray*}}
\newcommand{\pa}{\partial}
\newcommand{\la}{\lambda}
\newcommand{\sgn}{\mathrm{sgn}}
\newcommand{\ve}{\varepsilon}
\newcommand{\ra}{\rightarrow}

\renewcommand\qedsymbol{$\blacksquare$}
\DeclareMathOperator{\Ker}{Ker}
\DeclareMathOperator{\M}{M}
\DeclareMathOperator{\PC}{PC}
\DeclareMathOperator{\R}{r}
\DeclareMathOperator{\rA}{rA}
\DeclareMathOperator{\rB}{rB}
\DeclareMathOperator{\valA}{valA}
\DeclareMathOperator{\valB}{valB}
\DeclareMathOperator{\valC}{valC}
\DeclareMathOperator{\valE}{valE}
\DeclareMathOperator{\valP}{valP}


\newlength{\remainingwidth}
\setlength{\remainingwidth}{\textwidth-3cm-24pt}

\begin{document}

\begin{enumerate}
	\setcounter{enumi}{1}
	\item The code snipped given requires 3 stalls without forwarding.  This is
	because the second instruction, \texttt{mrmovl}, has no dependency
	on the first instruction, \texttt{addl}, as the first instruction only reads
	\texttt{\%edx}, and is done reading it in the decode stage.  However, the third
	instruction, \texttt{addl} requires three stalls, as it has to wait for
	\texttt{mrmovl} to write back to \texttt{\%edx}.  By that time, the
	first \texttt{addl} has finished writing to \texttt{\%eax}, \texttt{addl}'s
	other data dependency.  
	
	With forwarding, this code requires 1 stall.  This is because of a load/use
	hazard on the last two instructions, which cannot be resolved with forwarding
	completely.  
\end{enumerate}
\end{document}