\documentclass[12pt]{article}

% Useful packages that enable things like math symbols and a fancy header
\usepackage{amssymb,amsfonts,amsmath,amsthm}
\usepackage{latexsym}
\usepackage{mathrsfs}
\usepackage{fancyhdr}
\usepackage{outlines}
\usepackage{enumitem}
\usepackage{minted}

% Uses the package declared above and makes the page have headers and footers
\pagestyle{fancy}

% Puts a title at the top left
\lhead{CIS 314}
% Puts a title at the center
\chead{Assignment 4}

% Puts a title at the top right
\rhead{Mel McCalla}

% Put's "today's date" in the lower left with the command \today
\lfoot{\today}
\cfoot{\thepage}

%%%%  Some user defined commands that make life easier.  
\newcommand{\be}{\begin{equation}}
\newcommand{\ee}{\end{equation}}
\newcommand{\bea}{\begin{eqnarray}}
\newcommand{\eea}{\end{eqnarray}}
\newcommand{\beas}{\begin{eqnarray*}}
\newcommand{\eeas}{\end{eqnarray*}}
\newcommand{\pa}{\partial}
\newcommand{\la}{\lambda}
\newcommand{\sgn}{\mathrm{sgn}}
\newcommand{\ve}{\varepsilon}
\newcommand{\ra}{\rightarrow}

\renewcommand\qedsymbol{$\blacksquare$}
\DeclareMathOperator{\Ker}{Ker}

\begin{document}
\begin{enumerate}
	\item
	\begin{enumerate}
		\item
	\end{enumerate}
\end{enumerate}
\begin{minted}[tabsize=4]{gas}
	movl 8(%ebp), %esi  ; %esi holds x
	movl 12(%ebp), %ebx ; %ebx holds n
	movl $-1, %edi      ; Set result to -1
	movl $1, %edx       ; Set mask to 1
.L2:                    ; Beginning of for loop
	movl %edx, %eax     ; Load mask into %eax
	andl %esi, %eax     ; AND mask with x
	xorl %eax, %edi     ; XOR result with %eax
	movl %ebx, %ecx     ; Load n into %ecx
	shll %ecx, %edx     ; Shift n left by mask
	cmpl $0, %edx       ; Compare mask with 0
	jne .L2             ; Jump to .L2 if mask is 
                        ; not equal to 0
	movl %edi, %eax     ; Move result into %eax for returning
\end{minted}

\quad\\

\begin{enumerate}
	\item
	\begin{enumerate}
		\setcounter{enumii}{1}
		\item
	\end{enumerate}
\end{enumerate}
	\begin{minted}[tabsize=4]{c}
	int loop(int x, int n) {
		int result = -1;
		for (int mask = 1; mask != 0; mask = n << mask) {
			result ^= mask & x;
		}
		return result;
	}
	\end{minted}

\newpage

\begin{enumerate}
	\setcounter{enumi}{1}
	\item
	\begin{enumerate}
		\item
	\end{enumerate}
\end{enumerate}
\begin{minted}[tabsize=4]{gas}
.L3:                         ; Beginning of inner loop.
	movl (%ebx), %eax        ; Load *(A + M*i + j) into %eax
	movl (%esi,%ecx,4), %edx ; Load *(A + M*j + i) into t 
	movl %eax, (%esi,%ecx,4) ; Load *(A + M*i + j) into 
                             ; *(A + M*j + i) via %eax

	addl $1, %ecx            ; Add 1 to j
	movl %edx, (%ebx)        ; Load t into *(A + M*i + j)
	addl $52, %ebx           ; Add 1 to i
	cmpl %edi, %ecx          ; Compare j with M
	jl .L3                   ; Jump back to .L3 if j is less than M
\end{minted}

\begin{enumerate}
	\setcounter{enumi}{1}
	\item
	\begin{enumerate}
		\setcounter{enumii}{1}
		\item M is equal to 13
	\end{enumerate}
\end{enumerate}

\newpage

\begin{enumerate}
	\setcounter{enumi}{2}
	\item
\end{enumerate}
\begin{minted}[tabsize=4]{gas}
f:    
	movl 4(%esp), %eax       ; Load 4 bytes from the stack at the 
                             ; location 4 bytes forward from the 
                             ; stack pointer into %eax
	leal (%eax,%eax,2), %eax ; Calculate %eax+2*%eax
	ret                      ; Return by popping the return address 
                             ; off the stack and jumping there.  
                             ; %eax is conventionally the register 
                             ; that holds the return value, 
                             ; so this function returns x*3


g:
	movl 4(%esp), %edx       ; Load 4 bytes from the stack at the 
                             ; location 4 bytes forward from the 
                             ; stack pointer into %edx

	movl 8(%esp), %eax       ; Load 4 bytes from the stack at the 
                             ; location 8 bytes forward from the 
                             ; stack pointer into %eax

	leal (%eax,%eax,2), %eax ; Calculate %eax+2*%eax
	leal (%edx,%edx,2), %edx ; Calculate %edx+2*%edx
	addl %edx, %eax          ; Add %edx to %eax
	ret                      ; Return by popping the return address 
                             ; off the stack and jumping there.  
                             ; %eax is conventionally the register 
                             ; that holds the return value, so 
                             ; this function returns a*3 + b*3
\end{minted}

\end{document}
